\documentclass[10pt]{beamer}
\usetheme{Madrid}
\usepackage{hyperref}
\usepackage{color}
\usepackage{amssymb}
\usepackage{pifont}
%\usepackage{graphicx}
%\setbeamertemplate{itemize item}{\scriptsize\raise1.25pt\hbox{\donotcoloroutermaths$\blacktriangleright$}}
\setbeamertemplate{itemize item}{\ding{224}}
\setbeamertemplate{itemize subitem}{\scriptsize \ding{117}}
\setbeamertemplate{itemize subsubitem}{\tiny \raise1.25pt\hbox{\ding{107}}}
%%%%%%%%%%%%%%%%%%%%%%%%%%%%%%%%%%%%%%%%%%
%%%%%%%%%%%%%%%%%%%%%%%%%%%%%%%%%%%%%%%%%%
%%%%%%%%%%%%%%%%%%%%%%%%%%%%%%%%%%%%%%%%%%
\subtitle[(Statistical Genetics] {(Fundamentals) of Statistical Genetics}
\title{Module 15 - GWAS and Beyond}
\author{Lei Sun}
\date{}
\institute[University of Toronto]
{Department of Statistical Sciences, FAS \\Division of Biostatistics, DLSPH\\ University of Toronto} 
%%%%%%%%%%%%%%%%%%%%%%%%%%%%%%%%%%%%%%%%%%
%%%%%%%%%%%%%%%%%%%%%%%%%%%%%%%%%%%%%%%%%%
%%%%%%%%%%%%%%%%%%%%%%%%%%%%%%%%%%%%%%%%%%
\begin{document}
%%%%%%%%%%%%%%%%%%%%%%%%%%%%%%%%%%%%%%%%%%
\frame{ \titlepage}
%%%%%%%%%%%%%%%%%%%%%%%%%%%%%%%%%%%%%%%%%%
\begin{frame}
\frametitle{Outline}


\begin{itemize}
\item {\bf Chapter 11 - Genome-wide Association Studies (GWAS)}: the WTCCC example
\bigskip
\item {\bf Chapter 10 - Advanced Topics}: R demo of multiple hypothesis testing 
\bigskip
\item {\bf Chapter 12 - Looking Toward the Future}: some brief discussions.
\end{itemize}

\end{frame}
%%%%%%%%%%%%%%%%%%%%%%%%%%%%%%%%%%%%%%%%%%
\begin{frame}
\frametitle{Chapter 11 - GWAS}


\begin{itemize}
\item We will use the \href{http://www.nature.com/nature/journal/v447/n7145/full/nature05911.html}
{\textcolor{blue}{``first" GWAS study conducted by the Wellcome Trust Case Control Consortium (WTCCC)}} 
as an example.   
\bigskip
\item See \alert{separate notes/slides} prepared back in 2007 discussing the WTCCC GWAS in the 
\href{http://www.stage.utoronto.ca/home/smg}{\textcolor{blue}{SMG}} journal club and research seminar.  
Practices/issues fortunately/unfortunately have not changed much!
\end{itemize}
\end{frame}
%%%%%%%%%%%%%%%%%%%%%%%%%%%%%%%%%%%%%%%%%%
\begin{frame}[allowframebreaks]
\frametitle{Chapter 10/12 - Advanced Topics/Looking Toward the Future}

\begin{itemize}
\item The multiple hypothesis testing problem: 
%\href{http://www.utstat.toronto.edu/sun/StatGene-R/R-statgene-module-17-testing.R} {\textcolor{blue}{R demo.}}
\smallskip
\item[] For graduate students: 
\href{http://www.utstat.toronto.edu/sun/multitesting-index.html}
{\textcolor{blue}{STA4515-Multiple Hypothesis Testing and Its Applications}}
\bigskip

\item Monte Carlo methods: we have seen various examples in posted R codes.
\bigskip

\item We will not be able to discuss
\begin{itemize}
\item Haplotype analysis
\item Interaction model

\item Gene-based analysis: joint analysis of multiple SNPs
\item Pleiotropy: joint analysis of multiple phenotypes.
\item Pathway analysis
\item Meta-analysis with or without heterogeneity
\item Measurement errors or uncertainties
\item Longitudinal data (correlated measurements)
\item Design issues: population vs. family, single-stage vs. multi-stage etc.

\item Next generation sequencing
\item Different `omic' data: microarray, methylation, function annotation etc.
\item Data integration
\item Many more ...
\end{itemize}
\bigskip

% BIG data: heterogeneity issue, multiple testing, small p-values, tail distribution not accurate
\item Big data and recall Tim Harford's \href{http://onlinelibrary.wiley.com/doi/10.1111/j.1740-9713.2014.00778.x/epdf}{\textcolor{blue}
{Big data: are we making a big mistake?}}. 
\begin{itemize}
{\scriptsize \it 
\item While big data promise much to scientists, entrepreneurs and governments, they are doomed to disappoint us if we ignore some very \textcolor{red}{familiar statistical lessons}.
\item There are a lot of \textcolor{red}{small data problems} that occur in big data.  They don't disappear because you�ve got lots of the stuff.  They get worse.  
%GWAS 100K samples for 10M SNPs.
\item Data contain \textcolor{red}{systematic biases} and it takes careful thought to spot and correct
for those biases. Big data sets can seem comprehensive but the ``N = All" is often a seductive illusion.
\item The \textcolor{red}{multiple-comparisons problem}.  Statistical \textcolor{red}{correlation/patterns$\neq$Causation}.

\item Big data do not solve the problem that has obsessed statisticians and scientists for centuries: the problem of
insight, of \textcolor{red}{inferring} what is going on.  
\item Proving the value of statistics would also come from \textcolor{red}{interdisciplinary working}.  Teaming up with computer scientists, astronomers, the bioinformatics people. 
\item ``Big data" has arrived, but big insights have not. The challenge now is to solve new problems and gain new answers - without making the same old statistical mistakes on a grander scale than ever.
\item[]
}
\end{itemize}
\pagebreak

\item David Donoho's take on \href {http://pages.cs.wisc.edu/~anhai/courses/784-fall15/50YearsDataScience.pdf}
{\textcolor{blue}{Data Science}}

{\footnotesize \it 
\begin{itemize}

\item  \textcolor{red}{Data Science is statistics} 

\begin{center}
When physicists do mathematics, they don't say they're doing number science. They're doing
math. If you're analyzing data, you're doing statistics. You can call it data science
or informatics or analytics or whatever, but it's still statistics. ... You may not like
what some statisticians do. You may feel they don't share your values. They may embarrass
you. But that shouldn't lead us to abandon the term ``statistics''. 
(\href{http://genetics.wisc.edu/Broman.htm}{\textcolor{blue}{Karl Broman}}, Univ. Wisconsin)
\end{center}
\smallskip

\item  \textcolor{red}{The activities of Greater Data Science are classified into 6 divisions}
\begin{enumerate}
\item Data Exploration and Preparation
\item Data Representation and Transformation
\item Computing with Data
\item Data Modeling
\item Data Visualization and Presentation
\item Science about Data Science
\end{enumerate}
\smallskip
\item In 2065, mathematical derivation and proof will not trump conclusions derived from state-of-the-
art empiricism.   \textcolor{red}{Instead of deriving optimal procedures under idealized assumptions within mathematical models}, we will rigorously measure performance by empirical methods, based on the entire scientific literature or relevant subsets of it.
\smallskip
\item (BUT,) I am not arguing for a demotion of mathematics. I personally believe that  \textcolor{red}{mathematics offers the only way to create true breakthroughs}. The empirical method is simply a method to avoid self deception and appeals to glamor.
\end{itemize}
}
\end{itemize}
\bigskip

\centerline{\bf \large The challenges are endless but so are the opportunities!}

\end{frame}
%%%%%%%%%%%%%%%%%%%%%%%%%%%%%%%%%%%%%%%%%%

\begin{frame}
\frametitle{What's Next}

\centerline{\bf Joint the Statistical Genetics community!}
\bigskip

\centerline{\bf Thank you for taking the course and hope you have enjoyed it!}
\bigskip

\centerline{\bf Make our voice heard via \alert{Course Evaluations!}}

\end{frame}
%%%%%%%%%%%%%%%%%%%%%%%%%%%%%%%%%%%%%%%%%%

\end{document}
%%%%%%%%%%%%%%%%%%%%%%%%%%%%%%%%%%%%%%%%%%
%%%%%%%%%%%%%%%%%%%%%%%%%%%%%%%%%%%%%%%%%%
%%%%%%%%%%%%%%%%%%%%%%%%%%%%%%%%%%%%%%%%%%
\begin{frame}[allowframebreaks]
\frametitle{}

\begin{itemize}

\end{itemize}
\end{frame}
%%%%%%%%%%%%%%%%%%%%%%%%%%%%%%%%%%%%%%%%%%
\begin{columns}
\begin{column}{2in}

\end{column}
\begin{column}{2in}

\end{column}
\end{columns}

\begin{figure}
\includegraphics[width=2in]{../Figure/Book-Laird-Figure1-3}
\end{figure}

\begin{figure}
\includegraphics[width=2in,angle=90,angle=90,angle=90]{../Figure/Genome-Cartoon}
\end{figure}
