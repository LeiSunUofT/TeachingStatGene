\documentclass[10pt]{beamer}
\usetheme{Madrid}
\usepackage{hyperref}
\usepackage{color}
\usepackage{amssymb}
\usepackage{pifont}
%\usepackage{graphicx}
%\setbeamertemplate{itemize item}{\scriptsize\raise1.25pt\hbox{\donotcoloroutermaths$\blacktriangleright$}}
\setbeamertemplate{itemize item}{\ding{224}}
\setbeamertemplate{itemize subitem}{\scriptsize \ding{117}}
\setbeamertemplate{itemize subsubitem}{\tiny \raise1.25pt\hbox{\ding{107}}}
%%%%%%%%%%%%%%%%%%%%%%%%%%%%%%%%%%%%%%%%%%
%%%%%%%%%%%%%%%%%%%%%%%%%%%%%%%%%%%%%%%%%%
%%%%%%%%%%%%%%%%%%%%%%%%%%%%%%%%%%%%%%%%%%
\subtitle[Statistical Genetics] {(Fundamentals of) Statistical Genetics}
\title{Module 14 - Family-Based Association}
\author{Lei Sun}
\date{}
\institute[University of Toronto]
{Department of Statistical Sciences, FAS \\Division of Biostatistics, DLSPH\\ University of Toronto} 
%%%%%%%%%%%%%%%%%%%%%%%%%%%%%%%%%%%%%%%%%%
%%%%%%%%%%%%%%%%%%%%%%%%%%%%%%%%%%%%%%%%%%
%%%%%%%%%%%%%%%%%%%%%%%%%%%%%%%%%%%%%%%%%%
\begin{document}
%%%%%%%%%%%%%%%%%%%%%%%%%%%%%%%%%%%%%%%%%%
\frame{ \titlepage}
%%%%%%%%%%%%%%%%%%%%%%%%%%%%%%%%%%%%%%%%%%
\begin{frame} [allowframebreaks]
\frametitle{Outline}


{\bf Chapter 9 - Association Analysis in Family Designs}
\smallskip
\begin{itemize}
\item Transmission/Disequilibrium Test (TDT)
\begin{itemize}
\item	Design and data
\item	Falk and Rubinstein�s TDT 
\item	Spielman�s classical TDT
\item Connection with McNemar�s test for matched-pair data
\end{itemize}
\smallskip
\item Family-Based Association Test (FBAT)
\begin{itemize}
\item	The basic idea
\item TDT as a special case of FBAT
\end{itemize}
\end{itemize}

\end{frame}
%%%%%%%%%%%%%%%%%%%%%%%%%%%%%%%%%%%%%%%%%%
\begin{frame} 
\frametitle{Transmission/Disequilibrium Tests (TDT) - Introduction}

\begin{itemize}

\item Family-based association study design. 
\smallskip

\item Basic principle: at a given marker locus,  a parent of an affected child passes on only one of its two allele  to the child.   {\bf The transmitted allele (case) and  the non-transmitted allele (control) are a perfect match!}
\bigskip

\item Design and data

\begin{itemize}

\item Sample of $n$ trios, each family has one affected child.
\smallskip
\item The marker is genotyped for all the individuals. 
\item[] (It may not be possible for study of
late age-of-onset diseases in which case the parental
genotypes are often not observed.)
\bigskip

\begin{figure}
\includegraphics[width=1in]{../Figure/Book-Laird-Figure-9-1}
\end{figure}
\end{itemize}

\end{itemize}
\end{frame}
%%%%%%%%%%%%%%%%%%%%%%%%%%%%%%%%%%%%%%%%%%
\begin{frame} [allowframebreaks]
\frametitle{TDT - Falk and Rubinstein}

\begin{itemize}
\item Falk CF, Rubinstein P (1987).
\href{http://www.ncbi.nlm.nih.gov/pubmed/3500674}
{\textcolor{blue}{Haplotype relative risks: an easy reliable way to construct a proper control sample for risk calculations.}}
Ann. Hum. Genet. 51:227-233.
\smallskip

\item Among the $4n$ parental alleles: $2n$ are transmitted (cases) 
and $2n$ are not transmitted (controls):
\bigskip

\begin{columns}
\begin{column}{2in}
\begin{figure}
\includegraphics[width=1.5in]{../Figure/Book-Laird-Figure-9-1}
\end{figure}
\end{column}
\begin{column}{2in}
\begin{center}
\begin{tabular}{c||c|c||c} \hline \hline
& $A$ & $a$& Total \\ \hline \hline
TR & $x$ & $2n-x$& $2n$ \\ \hline
NTR& $y$ & $2n-y$& $2n$ \\ \hline \hline
Total & x+y & 4n-x-y& 4n \\ \hline \hline
\end{tabular}
\end{center}
\end{column}
\end{columns}
\pagebreak

{\scriptsize
\begin{center}
\begin{tabular}{c||c|c||c} \hline \hline
& $A$ & $a$& Total \\ \hline \hline
TR & $x$ & $2n-x$& $2n$ \\ \hline
NTR& $y$ & $2n-y$& $2n$ \\ \hline \hline
Total & x+y & 4n-x-y& 4n \\ \hline \hline
\end{tabular}
\end{center}
}
\bigskip

\item Hypothesis of interest:
$$H_0: \pi_1 = \pi_2,$$
where $\pi_1=P(A|\mbox{ transmitted})$, and $\pi_2=P(A |\mbox{ not transmitted})$.
\bigskip

\item[] Test if the cell probabilities of the two binomials are
equal (test the homogeneity of the binomial distributions), $\hat \pi_1 = \frac{x}{2n}$ and $\hat \pi_2= \frac{y}{2n}$.
\smallskip

\item Test statistic
$$T_1= \frac{4n(x-y)^2}{(x+y)(4n-x-y)} \sim \chi_1^2$$
\bigskip

\item The above can be derived either from the normal approximation
to the test of two proportions:

$$\frac{\frac{x}{2n} - \frac{y}{2n}}
{\sqrt{\frac{x+y}{4n}(1-\frac{x+y}{4n})
(\frac{1}{2n}+\frac{1}{2n})}} \sim N(0,1)$$
\bigskip

or the $\chi^2$ approximation to the test of homogeneity:

$$\sum \frac {(O-E)^2}{E} \sim \chi_1^2$$
\pagebreak

\begin{columns}
\begin{column}{2in}
\begin{figure}
\includegraphics[width=1.5in]{../Figure/Book-Laird-Figure-9-1}
\end{figure}
\end{column}
\begin{column}{2in}
\begin{center}
\begin{tabular}{c||c|c||c} \hline \hline
& $A$ & $a$& Total \\ \hline \hline
TR & $x$ & $2n-x$& $2n$ \\ \hline
NTR& $y$ & $2n-y$& $2n$ \\ \hline \hline
Total & x+y & 4n-x-y& 4n \\ \hline \hline
\end{tabular}
\end{center}
\end{column}
\end{columns}
\bigskip

\item {\bf Is the test valid?}
$$T_1= \frac{4n(x-y)^2}{(x+y)(4n-x-y)} \sim \chi_1^2 \mbox{ } ?$$

{\small
\begin{itemize}

\item Assumption: transmitted ($x$) and  non-transmitted ($y$)
should be independent, i.e. the two binomial distributions
should be independent of each other.
\smallskip
\item {\bf However}, if let $n_{AA}$ be the number of parents who are homozygous
$AA$, then 
$$x=n_{AA}+ ... \mbox{ } \mbox{ and } \mbox{ } y=n_{AA}+...$$
\pagebreak

\item In fact, if a parent is homozygous at the marker, 
no information is gained from the knowledge of the transmitted allele.
\bigskip
\item Only among heterozygous parents, $Aa$, the proportion that
allele $A$ was transmitted to the affected offspring gives us the
information about association (as well as linkage), and the null expectation
is 1/2!
\bigskip

\item The test based on $T_1$ generally leads to conservative results (estimated variance is bigger than the true variance, so the test statistic is small than the correct one). 
\end{itemize}
}

\end{itemize}
\end{frame}
%%%%%%%%%%%%%%%%%%%%%%%%%%%%%%%%%%%%%%%%%%
\begin{frame} [allowframebreaks]
\frametitle{TDT}

\begin{itemize}

\item Chapter 9.1 of the Textbook.
\smallskip

\item Spielman RS, McGinnis RE, Ewens WJ (1993). 
\href{http://www.ncbi.nlm.nih.gov/pmc/articles/PMC1682161/}
{\textcolor{blue}{
Transmission test for linkage disequilibrium}}: the insulin gene region and insulin-dependent diabetes mellitus (IDDM).
Am. J. Hum. Genet. 52:506-516.
\bigskip

\item {\bf Alternative view}: focus on the \underline{$2n$ parental genotypes} rather
than \underline{$4n$ parental alleles}. 
\smallskip

\item Note that {\bf each genotype is a matched pair}: 
the transmitted allele is a case, and the other one is not
transmitted and is a control. 

\item Thus, we should set up a table that exhibits the pairing: 
\bigskip

{\small
\begin{center}
\begin{tabular}{c||c|c||c} \hline \hline
&\multicolumn{2}{c||}{\mbox{ } NTR allele\mbox{ } } & \\ \cline{2-3}
\mbox{ } TR allele \mbox{ } & $A$ & $a$& Total \\ \hline \hline
$A$ & $w$ & $x$& $w+x$ \\ \hline
$a$& $y$ & $z$ & $y+z$ \\ \hline \hline
Total & $w+y$ & $x+z$ & $2n$ \\ \hline \hline
\end{tabular}
\end{center}
}
\pagebreak

{\scriptsize
\begin{center}
\begin{tabular}{c||c|c||c} \hline \hline
&\multicolumn{2}{c||}{\mbox{ } NTR allele\mbox{ } } & \\ \cline{2-3}
\mbox{ } TR allele \mbox{ } & $A$ & $a$& Total \\ \hline \hline
$A$ & $w$ & $x$& $w+x$ \\ \hline
$a$& $y$ & $z$ & $y+z$ \\ \hline \hline
Total & $w+y$ & $x+z$ & $2n$ \\ \hline \hline
\end{tabular}
\end{center}
}
\smallskip

\item It's easy to see that 
\begin{itemize}
\item $w$ is the total \# of homozygous $AA$ parents
\item $z$ is the total \# of homozygous $aa$ parents
\item $x+y$ is the total \# of heterozygous $Aa$ parents, among which
\item[] $x$ is the \# of heterozygous $Aa$ parents who transmitted $A$
\item[] $y$ is the \# of heterozygous $Aa$ parents transmitted $a$
\end{itemize}
\smallskip

\item Connection with the previous table?
{\scriptsize
\begin{center}
\begin{tabular}{c||c|c||c} \hline \hline
& $A$ & $a$& Total \\ \hline \hline
TR & $x^*$ & $2n-x^*$& $2n$ \\ \hline
NTR& $y^*$ & $2n-y^*$& $2n$ \\ \hline \hline
Total & $x^*+y^*$ & $4n-x^*-y^*$& $4n$ \\ \hline \hline
\end{tabular}
\end{center}
}

\pagebreak


\item Hypothesis of interest:
$$H_0: \pi=\frac12,$$
\item That is, test if the proportion of heterozygous $Aa$ parents who
transmitted $A$, $\pi=P(A \mbox{ was transmitted} | Aa)$, is 1/2.
\smallskip

\item Test statistic
$$T_2=\frac{(x-y)^2}{(x+y)} \sim \chi_1^2$$
\smallskip

{\small
\item The above can be derived either from the normal approximation
to the test of a proportion:
$$x \sim Bino(x+y, 1/2)$$
$$ \frac{\frac{x}{x+y}-\frac12}{\sqrt{\frac12 \frac 12 \frac{1}{x+y}}} \sim N(0,1)$$
\smallskip

\item Or as a direct application of the McNemar's test for data collected from matched-pair designs.  Here each genotype is a matched pair:  the transmitted allele is a case, and the other one is not transmitted and is a control. 
}
\end{itemize}
\end{frame}
%%%%%%%%%%%%%%%%%%%%%%%%%%%%%%%%%%%%%%%%%%
\begin{frame}
\frametitle{Relationship Between the Two TDT Approaches}

\begin{columns}
\begin{column}{2in}
{\scriptsize
\begin{center}
\begin{tabular}{c||c|c||c} \hline \hline
&\multicolumn{2}{c||}{\mbox{ } NTR allele\mbox{ } } & \\ \cline{2-3}
\mbox{ } TR allele \mbox{ } & $A$ & $a$& Total \\ \hline \hline
$A$ & $w$ & $x$& $w+x$ \\ \hline
$a$& $y$ & $z$ & $y+z$ \\ \hline \hline
Total & $w+y$ & $x+z$ & $2n$ \\ \hline \hline
\end{tabular}
\end{center}
}
$$T_2=\frac{(x-y)^2}{(x+y)}$$
\end{column}
\begin{column}{2in}
{\scriptsize
\begin{center}
\begin{tabular}{c||c|c||c} \hline \hline
& $A$ & $a$& Total \\ \hline \hline
TR & $x*$ & $2n-x*$& $2n$ \\ \hline
NTR& $y*$ & $2n-y*$& $2n$ \\ \hline \hline
Total & x*+y* & 4n-x*-y*& 4n \\ \hline \hline
\end{tabular}
\end{center}
}
$$T_1= \frac{4n(x^*-y^*)^2}{(x^*+y^*)(4n-x^*-y^*)}$$
\end{column}
\end{columns}
\bigskip

\begin{itemize}

\item It's easy to see that 
$$x^*=w+x, \: \: y^*=w+y$$

\item So
$$T_1=\frac{(x-y)^2}{(x+y+2w)(x+y+2z)/4n}$$
\smallskip

\item One can show that if HWE holds in the population, then the tests are asymptotically the same.
\end{itemize}
\end{frame}
%%%%%%%%%%%%%%%%%%%%%%%%%%%%%%%%%%%%%%%%%%
\begin{frame}
\frametitle{TDT - Some Additional Considerations}

\begin{itemize}

\item Can we detect association via TDT if there is linkage but no LD?

\item Appendix C of the Textbook: {\it The TDT Tests for Both Linkage and Association} and page 141-142:
\smallskip
{\it It is a test for both linkage and association, in the sense that both linkage and association must be present in order
for the test to reject. Said another way, if either linkage or association is absent, the
test has no power to reject.}
\smallskip
\item Thus, {\bf TDT is robust to population stratificaiton!}
\bigskip

\item Need genotype data for the parents: it may not be possible for the study of late age-of-onset
disease; it's more expensive than the population-based association approaches (Section 9.2.2 of the Textbook).
\bigskip

\item The simple TDT approach uses trio family structure so it is not flexible. 
It looses power because it ignores certain information such as affection status of the parents, families with more than one offspring, unaffected offsprings, mode of ascertainment, effect of other covariates etc.  
\end{itemize}
\end{frame}
%%%%%%%%%%%%%%%%%%%%%%%%%%%%%%%%%%%%%%%%%%
\begin{frame}[allowframebreaks]
\frametitle{FBAT-Introduction}

\begin{itemize}

\item Section 9.2  of the Textbook: {\it Family Based Association Tests: FBAT}

\item[] Laird et al. (2000).
\href{http://www.ncbi.nlm.nih.gov/pubmed/11055368}{\textcolor{blue}{
Implementing a unified approach to family-based tests of association.}}
\smallskip

\item {\it Assuming parental genotypes are known, and that there may be multiple offspring
in the family, the expression for the general FBAT score statistic is given by}
$$U =\sum_{i=1,\ldots n, \: j=1, \ldots,n_i}T_{ij}  \: (X_{ij}-E(X_{ij} |P_i)),$$
where 
{\small
\begin{itemize}
\item $i$ indexes families, 
\item $j$ indexes offspring in the $i_{th}$ family, 
\item $X_{i j}$ denotes the coded genotype
\item $T_{i j}$ denotes the coded trait value $Y$ of the $j_{th}$ offspring in the $i_{th}$ family
\item $P_i$ denotes the parental genotypes, and 
\item $n_i$ is the number of offspring in the $i_{th}$ family.
\end{itemize}
}
\item To obtain the FBAT test, we normalize $U$ by its standard deviation, again computed
under the conditional distribution of offspring genotype given $P_i$, giving a $Z$
or a $\chi^2_1$ statistic as
$$Z = \frac{U}{\sqrt{Var(U)}}, \: \mbox{ or } \: T = Z^2 = \frac{U^2}{Var(U)},$$
where computation of $var(U)$ depends upon whether or not linkage is present when
there are multiple siblings in a family. If we assume $\theta  = 1/2$ under $H_0$, then 
$$Var(U)=\sum_{i=1,\ldots,n,\: j=1,...,n_i}T_{ij}^2 \:  Var(X_{ij}|P_i)$$

\end{itemize}
\end{frame}
%%%%%%%%%%%%%%%%%%%%%%%%%%%%%%%%%%%%%%%%%%
\begin{frame}[allowframebreaks,fragile]
\frametitle{TDT as FBAT}
% Add the homework for TDT as FBAT.

\begin{itemize}
\item TDT is a special case of FBAT (Section 9.3.1 of the Textbook)

\begin{itemize}

\item TDT uses trio design, so $n_i=1$, only one layer of summary.
\smallskip

\item TDT uses only affected offsprings, so $T_{i}=1$,  and the statistics are simplified as

$$U =\sum_{i=1,\ldots, n}(X_i-E(X_i |P_i)),$$
\smallskip

$$Var(U)=\sum_{i=1,\ldots, n}Var(X_i|P_i)$$
\pagebreak

\item We need to derive the probability of an offspring genotype given parental genotype.  But we have learned this already! 
See Table 2.1 of the Textbook or the following simplified table which we have seen before:

{\scriptsize
\begin{verbatim}
               
  6 parental    Offspring Genotype   
  mating type    (in probability)       
                 AA     Aa    aa        
   AA x AA        1      0     0            
   AA x Aa      1/2    1/2     0            
   AA x aa        0      1     0            
   Aa x Aa      1/4    1/2   1/4         
   Aa x aa        0    1/2   1/2          
   aa x aa        0      0     1            

\end{verbatim}
}

\item The calculation of 
$$P(X|P)$$
would depend on the coding of $X$, e.g. Table 9.2 of the Textbook for the recessive model.
\pagebreak

{\small
\item Let's try it systematically for the recessive model of Textbook Table 9.2}
\end{itemize}
\end{itemize}
{\scriptsize
\begin{center}
\begin{tabular}{c|ccc|cc|ccc} \hline \hline 
Parental genotype&aa	&Aa	&AA     & X=0 &X=1 & $E(X|P)$&$E(X^2|P)$&$Var(X|P)$ \\ \hline
AA AA &	0  & 0  &1    &  0 & 1 & 1 & 1 &0 \\ \hline
AA Aa&	0  &1/2 &1/2  & 1/2 & 1/2 & 1/2& 1/2 & 1/4 \\ \hline
AA aa&	0  &	1 &  0&   1 & 0 & 0 & 0 & 0 \\ \hline
Aa Aa&    1/4&1/2 &1/4  & 3/4 & 1/4&  1/4 & 1/4 & 3/16 \\ \hline
Aa aa&    1/2&1/2 & 0  & 1 & 0 & 0 & 0 & 0 \\ \hline
aa aa&	1  & 0   & 0 & 1 & 0 & 0 & 0 & 0 \\ \hline \hline
\end{tabular} 
\end{center}
}


\begin{itemize}
\item[]
\begin{itemize}
{\scriptsize
\item Only $AA, Aa$ and $Aa, Aa$ mating types have variability, thus provide information.
\item Let $n_{AA, Aa, 0}$ be the trios such that mating type was $AA, Aa$ and the heterozygous parent $Aa$ transmitted $a$.
Similarly define $n_{AA, Aa, 1}$. 
\item[] And Let $n_{Aa, Aa, 1}$  be the trios such that mating type was $Aa, Aa$ and both parents transmitted $A$, $n_{Aa, Aa, 0}$ would be the remaining trios.
$$U =\sum_{i=1,\ldots, n}(X_i-E(X_i |P_i))$$
$$=n_{AA, Aa, 0} (0-1/2)+n_{AA, Aa, 1} (1-1/2)+  n_{Aa, Aa, 0}(0-1/4) + n_{Aa, Aa, 1} (1-1/4)$$
$$Var(U)=\sum_{i=1,\ldots, n}Var(X_i|P_i)=\frac14n_{AA, Aa} + \frac{3}{16} n_{Aa, Aa}$$
}
\end{itemize}
\end{itemize}
\end{frame}
%%%%%%%%%%%%%%%%%%%%%%%%%%%%%%%%%%%%%%%%%%
\begin{frame}
\frametitle{FBAT Software}
\begin{itemize}

\item \href{http://www.hsph.harvard.edu/fbat/default.html}{\textcolor{blue}{ Family-Based Association Test Toolkit}} (Both Windows and Linux versions!)
\bigskip


\item FBAT features include:

\begin{itemize}
{\footnotesize \it
\item Uses data from nuclear families, sibships, pedigrees, or any combination; provides unbiased tests with or without founder genotypes.
\item    Analyzes dichotomous, measured, or time-to-onset traits and multiple traits; trait definition can be optimized.
 \item   Analyzes markers on the x-chromosome with same options as autosomal marker analysis
  \item  Offers bi-allelic and multi-allelic tests of association using standard genetic models (additive, dominant, recessive or genotype).
  \item  Offers large sample and Monte-Carlo exact tests of the null hypothesis: no linkage and no association; offers large sample test of H0: no association.
 \item   Estimates allele frequencies; checks Mendelian consistency.
 \item   Tests multiple markers using haplotypes; estimates haplotype frequencies and linkage disequilibrium between pairs of markers.
  \item  Offers three multi-marker tests for testing multiple markers simultaneously, without resolving phase or assuming no recombination.
 \item   Offers two multiple trait tests.
  \item  Provides weighted or unweighted tests for rare variants, with empirical variance option.
  \item  Interactive and command driven program using standard pedigree data files; phenotype file is optional.
}
\end{itemize}

\end{itemize}
\end{frame}

%%%%%%%%%%%%%%%%%%%%%%%%%%%%%%%%%%%%%%%%%%

\begin{frame}
\frametitle{Exercises}

\begin{itemize}



\item Chapter 9 Exercise 1.

\item Chapter 9 Exercise 2.

\item Chapter 9 Exercise 3.

\item Chapter 9 Exercise 4.

\item Chapter 9 Exercise 5.



\end{itemize}
\end{frame}
%%%%%%%%%%%%%%%%%%%%%%%%%%%%%%%%%%%%%%%%%%
\begin{frame}
\frametitle{What's Next}

\begin{itemize}
\item Chapter 11 - Genome-wide Association Studies (GWAS)
\item Chapter 10 - Advanced Topics
\item Chapter 12 - Looking Towards the Future
\end{itemize}

\end{frame}
%%%%%%%%%%%%%%%%%%%%%%%%%%%%%%%%%%%%%%%%%%

\end{document}
%%%%%%%%%%%%%%%%%%%%%%%%%%%%%%%%%%%%%%%%%%
%%%%%%%%%%%%%%%%%%%%%%%%%%%%%%%%%%%%%%%%%%
%%%%%%%%%%%%%%%%%%%%%%%%%%%%%%%%%%%%%%%%%%

\begin{frame}[allowframebreaks]
\frametitle{}

\begin{itemize}

\end{itemize}
\end{frame}
%%%%%%%%%%%%%%%%%%%%%%%%%%%%%%%%%%%%%%%%%%
\begin{columns}
\begin{column}{2in}

\end{column}
\begin{column}{2in}

\end{column}
\end{columns}

\begin{figure}
\includegraphics[width=2in]{../Figure/Book-Laird-Figure1-3}
\end{figure}

\begin{figure}
\includegraphics[width=2in,angle=90,angle=90,angle=90]{../Figure/Genome-Cartoon}
\end{figure}
